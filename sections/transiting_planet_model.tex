%\subsection{The transiting planet model \label{transit_model}}
To model the starspot features within the transit portions of the light curve, we must explicitly model how the transits of the planet affect the amount of flux that reaches the observer. The visibility of the transits is comprised by the portion of the star that the planet occludes during transit and limb darkening effects. As discussed in Section~\ref{vis}, the planet will only transit along the boxes contained within the latitudes at $cos^{-1}(R_p \pm b)$. 
 
The first step in determining the visibility occluded by a planet at a given time step involves geometry which will be extended to ten cases contained in Table~\ref{cases}. The simplifying approximation that each of the boxes is a cartesian rectangle when it is projected onto the surface of the star is made. The midpoints of the edges of the cartesian rectangle are the same as the midpoints of the projected box edges. The top and bottom latitude limits of the cartesian rectangles are the same as those of the projected rectangles. Although this is not entirely accurate, it is a good approximation.  When a cartesian rectangle overestimates a portion of the projected box, it must also underestimate a similar but not necessarily equal portion of the box. Near the limbs, the curvature of the projected boxes as compared to the cartesian rectangles becomes more extreme, however the cartesian rectangles become smaller. In these cases the relative area misappropriated may be larger, the absolute error is smaller.

The calculations for the eclipsed visibility geometry are shown in full in ~\ref{trans_appendix}.

