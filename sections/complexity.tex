\externaldocument{tech_eclipse_text}

\subsection{Testing Complexity \label{complexity}}
The ability of the program to solve for brightness values has only been tested in two parameter spaces, complexity of the spots and noise of the light curve. The noise of the curve is generated by a python script based on Kepler Magnitude noises. Models were tested for a noiseless, a 12th magnitude star, and a 14th magnitude star. There is no noticeable difference between these.
The complexity of the spots is determined by the number of regions that vary from the default brightness value of 1.0. The model has been tested on 11 different spot configurations. These are one box, one box and one stripe, two boxes, two boxes and one stripe, two boxes and two stripes, three boxes and one stripe, three boxes and two stripes, three boxes and three stripes, four boxes and three stripes, all boxes randomly and two stripes, and all regions randomly darkened.