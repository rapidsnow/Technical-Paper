\externaldocument{tech_eclipse_text}

\section{Appendix \label{appendix}}
\begin{figure}[h]
	\centering
	\includegraphics{images/1b/14_noise/page_plot.png}
	\caption{Diagnostic plots for synthetic starspot system recovery. The top two plots show the brightness map as it appears on a projected star. The left map shows the input brightness values to the system for the production of the light curves. The map on the right shows the average recovered brightness over 24 windows that were modeled over the 30 day data set. Each window was 8 days wide and the increment between start times of the windows was 0.5 days. The next two plots show the box and stripe brightness values as a function of window. The region number corresponds to the tuple defined in Section~\ref{model_flux}, Equation~\ref{tuple}. The shade represents the brightness value. Each column in these plots is one window. The far right column (separated by a line) shows the input brightness value for that region. The next plot shows the standard deviation of the average recovered value in a region compared to the input value for that region. The boxes are to the left of the black line, and the stripes are to the right. Note that the region numbers are the same as in the box and stripe brightness plots above. The final plot is the fit to the synthetic light curve. This plot shows the entire month of data that was created. Every model is over plotted. The model light curves are shown in black while the synthetic light curves are the red points. Note that models overlap significantly due to the fact that the start times are only 0.5 days (~$\frac{1}{3}$ orbital phase units) apart per window while the window length is 8 days.}
	\label{page_1b}
\end{figure}
\begin{figure}[h]
	\centering
	\includegraphics{images/1b_1s/14_noise/page_plot.png}
	\caption{Diagnostic Plots}
	\label{page_1b_1s}
\end{figure}
\begin{figure}[h]
	\centering
	\includegraphics{images/2b/14_noise/page_plot.png}
	\caption{Diagnostic Plots}
	\label{page_2b}
\end{figure}
\begin{figure}[h]
	\centering
	\includegraphics{images/2b_1s/14_noise/page_plot.png}
	\caption{Diagnostic Plots}
	\label{page_2b_1s}
\end{figure}
\begin{figure}[h]
	\centering
	\includegraphics{images/2b_2s/14_noise/page_plot.png}
	\caption{Diagnostic Plots}
	\label{page_2b_2s}
\end{figure}
\begin{figure}[h]
	\centering
	\includegraphics{images/3b_1s/14_noise/page_plot.png}
	\caption{Diagnostic Plots}
	\label{page_3b_1s}
\end{figure}
\begin{figure}[h]
	\centering
	\includegraphics{images/3b_2s/14_noise/page_plot.png}
	\caption{Diagnostic Plots}
	\label{page_3b_2s}
\end{figure}
\begin{figure}[h]
	\centering
	\includegraphics{images/3b_3s/14_noise/page_plot.png}
	\caption{Diagnostic Plots}
	\label{page_3b_3s}
\end{figure}
\begin{figure}[h]
	\centering
	\includegraphics{images/4b_3s/14_noise/page_plot.png}
	\caption{Diagnostic Plots}
	\label{page_4b_3s}
\end{figure}
\begin{figure}[h]
	\centering
	\includegraphics{images/all_b_2s/14_noise/page_plot.png}
	\caption{Diagnostic Plots}
	\label{page_all_b_2s}
\end{figure}
\begin{figure}[h]
	\centering
	\includegraphics{images/all_varied/14_noise/page_plot.png}
	\caption{Diagnostic Plots}
	\label{page_all_varied}
\end{figure}