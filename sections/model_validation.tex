\externaldocument{tech_eclipse_text}

\section{Validating our model \label{validation}}
Extracting a two dimensional brightness map from a one dimensional light curve can result in degenerate solutions.  There is no way of knowing {\it a priori} what the brightness values of a real system should be at any given time and it is therefor impossible to verify our model with real data. To remedy this, we produce synthetic light curves using \fmod defined in Section~\ref{flux}. This technique will be described in greater detail below. 

While trying to verify the models, the only things that can vary for the reproduction is the number of boxes and stripes and the binning cadences. The number of stripes is chosen doing a sans-transit fit to the light curve with an increasing number of stripes until the chi-squared stops becoming significantly better. The number of boxes is chosen by calculating the relative area of the star that the planet occludes per time step. To resolve at a small scale, we want the planet to have three time steps in each box before moving on to the next one \citep{Huber2009}. Using these metrics, we chose the number of boxes and stripes for our system to be 11 and 22.  The in and out-of-transit binning cadences are chosen to mimic a real system. Because so many points exist out of transit and the overall trends tend to exist longer term a low cadence is desired for out-of-transit binning. For these example solves, the out-of-transit binning cadence is 95 Kepler short-cadence time steps per bin. The in-transit binning cadence is set at a higher frequency in order to remain sensitive to small scale variations in the light curves. For these example solves, the in-transit binning cadence is set to three Kepler short-cadence time steps (58.85 seconds) per bin.

%1) Choose a number of stripes and boxes. Define the tuple of boxes and stripes - Done
%2) To simulate spots, we darken certain regions. The average brightness of an undarkened star is 1.0 (Note, our synthetic models might actually be the issue. They are an unphysical system) - Done
%3) We use Kepler-17 parameters (b, a, epoch, Prot, Rp, orb_per, etc.).
%4) Rotate our simulated star and generate an integrated flux measurement at each time point according to \fmod using the visibilities produced by the C code in section~\ref{vis} and a predefined brightness set.
%5) Total duration of one set of Kepler data (one month).
%6) There is no evolution in the b values during this time.
%7) Add noise to the light curve. Give values for sigma and link to the webpage.