\externaldocument{tech_eclipse_text}

\section{Validating our model \label{validation}}
Theoretically, the model should reproduce the light curves with good precision, however degenracies exist in the solutions. Trying to extract a two dimensional brightness map from a one dimensional light curve is somewhat tricky.  That the model light curve fits the data and that the brightness values are correct must be verified. This cannot be done with real data. Only half of the necessary information can be proved correct. There is no way of knowing {\it a priori} what the brightness values of a real system should be at any given time. To remedy this, we produce synthetic light curves using \fmod   defined in Section~\ref{flux}. This technique will be described in greater detail below. 

While trying to verify the models, the only things that can vary for the reproduction is the number of boxes and stripes and the binning cadences. The number of boxes and stripes are set to six and eighteen respectively.  The in and out-of-transit binning cadences are chosen to mimic a real system. Because so many points exist out of transit and the overall trends tend to exist longer term a low cadence is desired for out-of-transit binning. For these example solves, the out-of-transit binning cadence is 95 Kepler short-cadence time steps per bin. The in-transit binning cadence is set at a higher frequency in order to remain sensitive to small scale variations in the light curves. For these example solves, the in-transit binning cadence is set to 3 Kepler short-cadence time steps per bin.