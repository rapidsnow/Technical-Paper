\documentclass[iop]{emulateapj}
\usepackage{graphicx,pdfpages,array,amsmath}
\usepackage{xr,natbib}

\externaldocument{sections/setup_and_terminology}
\externaldocument{sections/transiting_planet_model}
\externaldocument{sections/limb_darkening}
\externaldocument{sections/visibility}
\externaldocument{sections/model_flux}
\externaldocument{sections/model_validation}
\externaldocument{sections/model_lightcurves}
\externaldocument{sections/complexity}
\externaldocument{sections/results}
\externaldocument{bibliography_astro}

\newcommand{\fmod}{\mbox{$f_{mod,i}$}}
\newcommand{\chisq}{\mbox{$\chi^2$}}
\newcommand{\fobs}{\mbox{$f_{obs,i}$}}
\newcommand{\vij}{\mbox{$V_{ij}$}}
\newcommand{\bj}{\mbox{$B_{j}$}}
\newcommand{\sigi}{\mbox{$\sigma_{i}$}}

\begin{document}
\title{Eclipse Mapping Code}
\author{Woody Austin}

\nocite{*}

\begin{abstract}
pass
\end{abstract}
\maketitle

\bibliographystyle{plainnat}
\setcitestyle{author year,round,semicolon,aysep={}}

\linespread{0.9}
\selectfont

\setcounter{tocdepth}{2}
\addtocounter{page}{0}

\clearpage

% INTRO
%The program is designed to model exhibit variability 
%in their light curve due to starspots rotating in and out of view on the surface of the star as
%well as features

%Section 1 Model, then subsections until validating the model
%Put in all of the figures


\externaldocument{tech_eclipse_text}

\section{Set-up and Terminology \label{terminology}}
The eclipse mapping program described in this paper derives a model for the relative surface brightness of a transiting planet host star when given a single band, short cadence light curve of the target as an input. Each iteration of the program produces a static map of the stellar surface. By applying the code to a series of short segments of the light curve, we are able to extract information about the time evolution of the star's surface brightness \citep{Huber2009}.  

In order for the program to create a brightness map, certain physical properties of the star and the planet must be provided. Stellar rotation period ($P_{rot}$), limb-darkening coefficients, orbital period ($P_{orb}$), orbital separation ($a$), impact parameter ($b$), and transit duration are all required in order for the program to produce an adequate model of the physical system. The star is assumed to be a rotating, uniform, solid body. In addition, the spin-axis of the star is required to be aligned with the orbital axis of the planet. The planet must also be on an approximately circular orbit. These criteria are both met for our test object, Kepler-17 \citep{Borucki2011}.

%Possibly move this next paragraph or restructure it. 
Other inputs to the code include the binning cadence for both in and out-of-transit points of the light curve, the number of boxes ($n_b$) and stripes ($n_s$) there should be in the surface map, number of days per individual brightness solution or {\it window}, and the number of days to increment each window. In Figure~\ref{LC_Fit}, the window is the section with the red points.

\begin{figure}[h]
	\centering
	\includegraphics[width=.5\textwidth]{images/2b_2s/14_noise/model_fit_w15.png}
	\caption{Typical light curve fit produced by the Eclipse Mapping code. The green points are the overall light curve data, the red points are the data for the current window, and the black line is the model fit for that window.}
	\label{LC_Fit}
\end{figure}

The data is binned according to two separate binning cadences. In-transit points give information about smaller scale features, so they are binned at a high cadence. The out-of-transit points are binned at a low cadence because the overall longterm variability of a longitude can be tracked well with less information. Added benefits of binning the out-of-transit points at a low cadence are that the run-time of the code decreases with fewer points to match in the light curve and there is some degree of inherent noise removal by binning. Care must be taken at the boundaries of in- and out-of-transit regions of the light curve. Some bins will be cut off with fewer points in order to properly switch between the two different binning cadences.

In order to efficiently describe the details of our program, the geometry of the problem must be defined and some basic terminology must be established. The stellar surface is divided into a series of large {\it regions}, as shown in Figure~\ref{CoRoT}. While describing our program in this paper, the regions along the line of transit will be called {\it boxes}; the total longitudinal regions will be called {\it longitudes}; and the longitudinal regions with the box areas subtracted will be referred to as {\it stripes}. When talked about as an ensemble or when the distinction is unimportant, these regions will be referred to as just that - {\it regions}.

%\begin{figure}[h]
%	\centering
%	\includegraphics[width=.5\textwidth]{images/modelGeometry.png}
%	\caption{A typical setup for a brightness map produced by the code. \cite{Huber2009}}
%	\label{CoRoT}
%\end{figure}
\begin{figure}[h]
	\centering
	\includegraphics[width=.5\textwidth]{images/2b_2s/14_noise/brightness_map_w15.png}
	\caption{Typical setup for the Eclipse Mapping code. Note the boxes along the region that is eclipsed, and the overall longitudes.}
	\label{bright_map}
\end{figure}

Our program calculates the model flux at each time step, by summing over the surface of the visible sphere according to:

\begin{equation}
	\fmod = \sum_j V_{i,j}b_j, 
\end{equation}

where $b_j$ is the brightness per unit area for region $j$ and $V_{i,j}$ is the {\it visibility} of region $j$ at time, $i$. Visibility is a measure of projected surface area along the line of sight. Eclipses are included in the visibility. When the planet occludes part of the star, the relative visible area of the star decreases. The eclipse is accounted for by subtracting the area that the planet obscures of a given box from the normal visibility of the same box at the given time step. Done this way, the transit produced in the model light curve is a box-car transit shape. To give a more smooth and realistic transit curve, limb-darkening is included according to the quadratic limb darkening law provided in \citep{Claret2004}.

The Amoeba Algorithm is a minimization tool that works well for small numbers of dimensions (regions in this case) and which will always converge to some minima (although not necessarily absolute) \citep{NR}. It determines the brightness values given pre-defined visibilities as described in section~\ref{vis}. Both of these, brightness values and visibilities, apply to a set of $j = n_{boxes} + n{stripes}$ regions. The planet will occlude only the boxes and will do so only during a transit. The longitudes' brightness values are defined as:

\begin{equation}
%Z_j = \frac{S_j - \frac{c}{q} \sum_{j=1}^{q}B_j}{1- c}
S_j = Z_j (1 - c) + \frac{c}{q} \sum_{j=1}^{q}B_j
\label{z_val}
\end{equation}
%Changed the above equation... switched Z and S values.

Where $q$ is the ratio of $n_{boxes}$ to $n_{stripes}$ and $c$ is the ratio of the total eclipsed area to the non-eclipsed area. $c$ can be calculated by the same set of integrals that will be used for the visibilities. The Amoeba algorithm is given the set of box and longitude visibilites $\{b_1, ..., b_j, z_1, ..., z_n\}$. When the planet is not transiting, the only information available is about the total longitudinal brightness information. Within the chi-squared call of the Amoeba algorithm, the box and stripe visibilities and brightnesses are calculated as a way to blend the parameters together and get information about the overall longitude. This allows the use of boxes and stripes whose sum can be thought of as the longitude values in the Amoeba while still getting information about the boxes and the longitudes independently. This is called parameter interdependence. This appears to encourage the Amoeba Algorithm to navigate to a better local minima than if this process of creating parameter interdependence and using optimal variables were omitted. 

%The longitude values from equation~\ref{z_val} create parameter independence by changing the way that our algorithm navigates the chi-squared space so that such a case is less likely to become a local minimum. The Amoeba algorithm takes the set of box brightness guesses and longitude brightness guesses as inputs. During each call of the chi-squared algorithm the stripe values are calculated from the longitudes and boxes. The boxes and the stripes are then used to calculate the model flux.



















\vspace{9mm}
\externaldocument{tech_eclipse_text}

\subsection{Visibility Calculations of Regions \label{vis}}
Recall that in order to calculate \fmod, the visibility of each region at a given time step is required. These visibilities are calculated by finding the projected surface area of a sphere for a given set of angles. The basic integral is the spherical surface integral dotted with the unit vector, $\hat{x}$, along the line-of-sight:

\begin{equation}
	V_{i,j} = \int_{\phi_1}^{\phi_2} \int_{\theta_1}^{\theta_2} \sin^2{\theta}\cos{\phi}\,\mathrm{d}\theta \, \mathrm{d}\phi
\end{equation}

Where $\phi_1$, $\phi_2$, $\theta_1$, and $\theta_2$ are in standard spherical coordinates. For all types of visibilities, $\phi_1$ and $\phi_2$ are set by the number of stripes or boxes defined at the start of the program. The $\theta$ limits, however, will be different depending on region type.
The $\theta$ limits for the boxes are set by the impact parameter and the radius of the planet. The limits will be equal to $cos^{-1}(b \pm R_p)$. For the visibility of a longitude, $\theta$ will range from 0 to $\pi$. The visibility of a stripe is calculated (in the chi-squared routine) by subtracting the sum of the box visibilities in a given longitude range from the longitude in the same range. A typical box visibility curve is shown in Figure~\ref{box_vis}. %The sum of every region over time is shown in Figure~\ref{sum_vis}.

%\begin{figure}[h]
%	\centering
%	\includegraphics[width=.5\textwidth]{images/sum_vis.png}
%	\caption{Sum of visibilities over time}
%	\label{sum_vis}
%\end{figure}

\begin{figure}[h]
	\centering
	\includegraphics[width=.5\textwidth]{images/box_vis.png}
	\caption{Box visibility curve}
	\label{box_vis}
\end{figure}
\vspace{9mm}
\subsection{The transiting planet model \label{transit_model}}
To actually model the starspot features within the transit portions of the light curve, a reasonable understanding of the visibility profile of a transit for any given system is required. The visibility of the transits is determined by the portion of the star that the planet occludes during transit and limb darkening effects. As discussed earlier, the planet will only transit along the boxes contained within the latitudes at $cos^{-1}(R_p \pm b$. 
 
The first step in determining the visibility occluded by a planet at a given time step involves geometry which will be extended to the ten cases contained in Table~\ref{cases}. The simplifying approximation that each of the boxes is a cartesian rectangle when it is projected onto the surface of the star is made. The midpoints of the edges of the cartesian rectangle are the same as the midpoints of the projected box edges. The top and bottom of the cartesian rectangles are the same as the projected rectangles. Although this is not entirely accurate, it is a good approximation because when a cartesian rectangle overestimates a portion of the projected box, it must also underestimate a similar but not necessarily equal portion of the box. Near the limbs, the curvature of the projected boxes as compared to the cartesian rectangles becomes more extreme, however the cartesian rectangles become smaller so that although the relative area misappropriated may be larger, the absolute error is smaller.

The simplest case in which the planet is entirely contained within a box is:
\begin{equation}
	A_{occluded} = \pi r_p^2
\end{equation}

The planet can also be partially contained in a box in many ways, shown in Table~\ref{cases}. The sliver of the planet that is not within the box will be calculated and then extended to take care of all of the various cases. To do this, the area of a sector of a circle is found and then the triangle that is formed within it is subtracted. The sector is the combination of the light blue and yellow regions in Figure~\ref{eclipse}.
\begin{figure}[h]
	\centering
	\includegraphics[width=.5\textwidth]{images/figure.png}
	\caption{Geometry of eclipse path visibility.}
	\label{eclipse}
\end{figure}

\begin{equation}
	A_{sector} = \frac{r_p \alpha^2}{2}
\end{equation}

where:

\begin{equation}
	\alpha = 2 \cos^{-1}\left(\frac{r^{\prime}}{r_p}\right)
\end{equation}

The area of the triangle is $cr^{\prime}$. $c$ is one half the length of the chord in the figure. $c$ can be found with the Pythagorean theorem. Knowing this, the area of the triangle becomes: 

\begin{equation}
	A_{triangle} = r^{\prime}\sqrt{r_p^2 - {r^{\prime}}^2}
\end{equation}

Finally, the area of the segment that indicates the part of the planet just over the border of the box is:

\begin{equation}
	A_{seg} = r_p^2 \cos^{-1}\left(\frac{r^{\prime}}{r_p}\right) - r^{\prime} \sqrt{r_p^2 - {r^{\prime}}^2}
\end{equation}

Using $A_{seg} $ and Table~\ref{cases}, a believable box-car transit model is produced. Note that $A_{seg,l}$ refers to a segment that is off the left side of a box relative to the center of the planet and likewise for $A_{seg,r}$. The subscripts $A_{seg,r1}$ and $A_{seg,r2}$ and likewise for the left refer to cases where there are two portions off one side of a box relative to the center of the planet. Limb darkening must also be included in order to give a better approximation to the actual shape of a real transit. This is introduced in the visibility rather than in the model flux calculation. This is okay because $\fmod = V_{i,j} b{j}$. If limb-darkening were calculated during every flux calculation of every chi-squared call during the Amoeba algorithm run, it would require much more operational complexity in the code. However, limb-darkening can be calculated before the many chi-squared calls via the Amoeba algorithm and and avoid this.  The geometric analysis of the transit and limb darkening is continued in the next section.

\begin{table}
	\caption{Subcases for planetary eclipse visibility}
	\label{cases}
	\begin{center}
	\renewcommand{\arraystretch}{1.2}
		\begin{tabular}{| m{.04\textwidth} | m{.24\textwidth} | m{.15\textwidth} |} %c means center justify, l left, r right
			\hline
			\textbf{Case}    & \textbf{Description} & \textbf{Area covered by planet in box}\\ %Separate with &, lines with \\
			\hline%horizontal line
			I      &   Planet is completely out of the box to the right                                                                      & 0                                                           \\ \hline
			II     &   Planet is completely out of the box to the left                                                                         & 0                                                           \\ \hline
			III    &   Planet is completely contained in the box                                                                              & $\pi r_p^2$                                         \\ \hline
			IV    &   Planet is partially off the right side of the box. Center is inside of the box                       & $\pi r_p^2 - A_{seg,l}$                     \\ \hline
			V     &   Planet is partially off the right side of the box. Center is outside of the box                     & $A_{seg}$                                          \\ \hline
			VI    &   Planet is partially off the left side of the box. Center is inside of the box                          & $\pi r_p^2 - A_{seg,r}$                     \\ \hline
			VII   &   Planet is partially off the left side of the box. Center is outside of the box                        & $A_{seg}$                                          \\ \hline
			VIII  &   Planet is partially off both sides of the box. Center is inside of the box                            & $\pi r_p^2 - A_{seg,r} - A_{seg,l}$ \\ \hline
			IX    &   Planet is partially off the right side of the box. Center is outside of the box to the left & $A_{seg,l1} - A_{seg,l2}$                \\ \hline
			X     &   Planet is partially off the left side of the box. Center is outside of the box to the right    & $A_{seg,r1} - A_{seg,r2}$               \\ \hline
		\end{tabular}
	\end{center}
\end{table}


\vspace{9mm}
\externaldocument{setup_and_terminology.tex}
\externaldocument{transiting_planet_model.tex}
\externaldocument{limb_darkening.tex}
\externaldocument{visibility.tex}
\externaldocument{model_flux.tex}
\externaldocument{model_validation.tex}
\externaldocument{model_lightcurves.tex}
\externaldocument{complexity.tex}
\externaldocument{results.tex}

\section{Limb Darkening}
To actually calculate the limb-darkening, a quadratic limb-darkening law is used.
\begin{equation}
   \frac{I(\psi)}{I(0)} = 1 - c_1 (1 - \mu) - c_2 (1 - \mu)^2
\end{equation}
Here, $\mu$ is the cosine of the angle between the line of sight and the normal at the point at which you are trying to calculate the limb darkening. Because stars are so distant, this can be approximated as $\hat{x} = R^* \sin{\theta}\cos{\phi}$, but $R^{*}$ is defined as 1.0 throughout all calculations.

To complete the limb darkening model, the average value in one of the regions is calculated at every timestep. This is done analytically from the integral:

\begin{equation}
\begin{split}
    \frac{1}{V_{i,j}} \int_{\phi_1}^{\phi_2} & \int_{\theta_1}^{\theta_2}  (1 - c_1 (1 - \sin{\theta}\cos{\phi}) \\ &- c_2 (1 - \sin{\theta}\cos{\phi})^2) \sin{\theta}\cos{\phi}\,\mathrm{d}\theta \, \mathrm{d}\phi
\end{split}
\end{equation}

This works well for limb-darkening determination in each region, but care must be taken when calculating limb darkening for the eclipse. Rather than calculating equations of limb darkening for the eclipse directly as in Mandol \& Agol 2000, the ratio of the limb-darkened eclipsed region to the non-limb-darkened eclipsed region is used. If the number of regions in eclipse is not high enough, a boxy limb-darkening approximation to the transit path is produced using this scheme. This happens because the planet is moving too quickly relative to the surface of the star given the in-transit binning cadence. To fix this, the ratio of the limb-darkened to non-limb-darkened regions is calculated for multiple sub-regions. This gives a finer grain approximation to the real value.


\vspace{9mm}
\externaldocument{tech_eclipse_text}

\section{Model Flux \label{flux}}
%Note: This section will be moved out of its current location
The flux of the star at any given time $i$ is given by $\fmod = \sum{V_{i,j} b_{j}}$. Each region on the star has some visibility value $V_{i,j}$ that varies with time. These visibility values are all governed by the amount of projected area that is seen by the observer over time and can be calculated via an analytical integral. Finding the brightness values, $b_j$, is the goal of the program. For a given set of data the brightness values cannot change using our technique. To remedy this, each month (or quarter) of data is divided into smaller intervals called {\it windows}. By moving the start and end times of the windows by small amounts, information about the evolution of starspots can be inferred. Previous estimates on the timescale of starspot evolution r eport that it occurs on the order of 10-30 days. This works well with the model described here. The model does a better job of fitting smaller sections of data than larger ones.

The set of brightness values is optimized by minimizing the chi-squared:
\begin{equation}
	\chi^2 = \sum \frac{(\fmod - \fobs)^2}{\sigma_i^2}
\end{equation}

The Amoeba Algorithm is used to do the actual minimization. The algorithm runs quickly and will always converge given any set of inputs. The largest downside (of most optimization algorithms) is that it cannot distinguish between local and absolute minima. This becomes especially apparent when choosing an initial set of brightness guesses for the simplex structure. Because the average brightness of the star is 1.0, the initial conditions for the simplex structure are $n_{sb} + 1$ brightness sets of $n_{sb}$ values where $n_{sb} = n_{stripes} + n_{boxes}$ and every element is in the simplex is 1.0. A different (orthogonal) basis vector $e_i$ times a scale factor, $s$, is then added to each of the $n_{sb} + 1$ sets. When the scale factor is too large, the Amoeba converges on answers that are not likely real. When the scale factor is too small or non-existent, the Amoeba does not do any work as the relative difference between successive $\chi^2$ calculations is small despite having not converged on an actual minimum.

\vspace{9mm}
\externaldocument{tech_eclipse_text}

\section{Validating our model \label{validation}}
%Things that were in the version that got lost:
%	Rework first two sentences
%	3) Explain that we chose 11/22 as per the section before this


Extracting a two dimensional brightness map by fitting a model to a one dimensional light curve can result in degenerate solutions.  There is no way of knowing what the brightness values of a real system should be {\it a priori}. To remedy this, we produce synthetic light curves using our definition of model flux, a set of visibilities, and a known set of input brightness values. The average brightness on the surface of the star is defined to be 1.0. This is the default value for this known set of input brightness values. To simulate starspots, we darken regions by giving them a brightness value of less than 1.0. For example, if we want to simulate a group of spots in only one {\it box}, we would set all of the input brightness values to 1.0 except for one {\it box} that has the value 0.90. One the spots are darkened, the flux is then integrated at each time step over all of the {\it regions}. There are currently 11 synthetic light curves with different spot configurations that we test with our system, but only a subset will be detailed in the paper.

\begin{table}
\begin{center}
  \begin{tabular}{l | l}
    Parameter Name & Value \\ \hline
    Rotation Period & 11.89\\
	$\frac{R_p}{R^{*}}$ & 0.129530 \\
	Limb Darkening Coefficient 1 & 0.4282\\
	Limb Darkening Coefficient 2 & 0.2514\\
	Orbital Period & 1.485711\\
	Orbital Epoch & 352.678035\\
	Orbital Separation & 5.670\\
	Impact Parameter & 0.01800\\
	Transit Duration & 0.094775\\
  \end{tabular}
\end{center}
\label{models}
\end{table}

We want our synthetically produced light curves to mimic light curves from real systems. To ensure that this happens, we adopt the stellar, planetary, and orbital properties directly from the Kepler-17 system provided in \citet{Borucki????} ****Cite the correct parameter place and shown in Table~\ref{parameters}. These parameters are then used to create the visibilities, as described in Section~\ref{visibilities}, and determine an appropriate number of stripes and boxes for the system as described in Sections~\ref{visibilities} and ~\ref{flux}. For the Kepler-17 system and for our synthetic systems, we choose the number of {\it stripes} and {\it boxes} to be 11 and 22 respectively.

A typical spot on a real star is anywhere from 500 to 1000 degrees cooler than the rest of the stellar surface. This corresponds to about 0.67 the brightness of an average part of the star in the starspot \citep{Walkowicz2013}. The {\it regions} in our synthetic systems are designed to imitate a real star with at most 30\% spot coverage. Therefore, the brightness values in our {\it regions} vary from 1.0 down to 0.88 in the {\it boxes} (higher percentage of spot coverage) and down to 0.91 in the {\it stripes}. 

When the light curves are produced in this way, they do not have any noise. This is clearly a bad imitation of a real system. We introduce Gaussian noise with standard deviations based on the average photon error counts provided in the table at \citet{noise_levels_table}****Noise level table. We try to recover the brightness values for noiseless, 12th magnitude noise, and 14th magnitude noise synthetic light curves. We show examples of the same light curve with different levels of noise in Figure~\ref{noise_comp}. Kepler-17 itself is a fourteenth magnitude star.

\begin{figure}
	\centering
	\includegraphics[width=.5\textwidth]{images/noise_levels.eps}
	\caption{Comparison of noise of models. The noise for these models was based off of error counts similar to a given magnitude Kepler star. Our tests included models with no noise in blue, simulated 12th magnitude noise (441 counts per million) in red, and simulated 14th magnitude noise (1620 counts per million) in green \citep{noiseCounts}.}
	\label{noise_comp}
\end{figure}

The in and out-of-transit binning cadences also mimic how we would solve for a real system. Because so many points exist out of transit and the trends from larger, possibly polar spots tend to have longer timescales, a low binning cadence is desired for out-of-transit binning. For these example solves, the out-of-transit binning cadence is 95 Kepler short-cadence time steps (58.85 seconds) per bin. The in-transit binning cadence is set at a higher frequency in order to remain sensitive to small scale variations in the light curves due to small scale starspots in the path of the planet. For these example solves, the in-transit binning cadence is set to three Kepler short-cadence time steps per bin.

Our predetermined set of brightness guesses $B = \{b_1, b_2, ..., b_{nboxes}, b_{stripes}, ..., b_j\}$ remains constant over then entire period of data in question (usually a month of Kepler short-cadence data). This is not a good approximation to real data as it does not include any notion of starspot evolution. However, over the short period of each {\it window} the evolution is not a noticeable factor, and it makes more sense to produce the lightcurves in this way.



%While trying to verify the models, the only things that can vary for the reproduction is the number of boxes and stripes and the binning cadences. The number of stripes is chosen doing a sans-transit fit to the light curve with an increasing number of stripes until the chi-squared stops becoming significantly better. The number of boxes is chosen by calculating the relative area of the star that the planet occludes per time step. To resolve at a small scale, we want the planet to have three time steps in each box before moving on to the next one \citep{Huber2009}. Using these metrics, we chose the number of boxes and stripes for our system to be 11 and 22. 
\vspace{9mm}
\externaldocument{tech_eclipse_text}

\subsection{Producing the model light curves \label{modelLC}}
To produce the model light curves, the model flux $\fmod = V_{i,j} b_j$ was used. The visibility calculations are the same as described in section~\ref{vis}. Rather than running the Amoeba algorithm to determine a best fit to an existing light curve, a predetermined set of brightness guesses $B = \{b_1, b_2, ..., b_j\}$ is used to produce a new light curve. The light curves that were used to test the code had stellar and planetary parameters similar to that of Kepler 17. $P_{rot}$, $P_{orb}$, $R_p$, orbital separation, limb darkening (temperature), and transit width are all identical to the Kepler 17 system. The light curves were produced using 6 stripes and 18 boxes. To simulate spots, certain regions are darkened more than the default brightness value of 1.0. When the light curves are produced, there is no noise. To remedy this, a simple python script generates Gaussian noise based on a Kepler-Magnitude input. 
\begin{figure}
	\centering
	\includegraphics[width=.5\textwidth]{images/noise.png}
	\caption{Comparison of noise of models.}
	\label{noise_comp}
\end{figure}

The noise for these models was based off of error counts similar to a given magnitude Kepler star. Our tests included models with no noise, simulated 12th magnitude noise, and simulated 14th magnitude noise. Figure~\ref{noise_comp} shows a {\it window} of our model light curve with the three levels of noise for comparison.
\vspace{9mm}
\externaldocument{tech_eclipse_text}

\subsection{Testing Complexity \label{complexity}}
The ability of the program to solve for brightness values has only been tested in two parameter spaces, complexity of the spots and noise of the light curve. The noise of the curve is generated by a python script based on Kepler Magnitude noises. Models were tested for a noiseless and 12th magnitude star. There is no noticeable difference between these.
The complexity of the spots is determined by the number of regions that vary from the default brightness value of 1.0. The model has been tested on 11 different spot configurations. These are one box, one box and one stripe, two boxes, two boxes and one stripe, two boxes and two stripes, three boxes and one stripe, three boxes and two stripes, three boxes and three stripes, four boxes and three stripes, all boxes randomly and two stripes, and all regions randomly darkened.

\vspace{9mm}
\externaldocument{tech_eclipse_text}

\subsection{Results \label{results}}
The program does a good job of recovering the input brightness values and matching the light curves. The level of noise has some effect on the reproduction of the brightness values, but clear trends can still be seen even with simulated 14th magnitude noise. One important thing to note is that the program does not recover the correct brightness values every time, but does a good job when averaged over many windows. This means that the results are better interpreted as long term trends than as exactly correct in every window. For the synthetic light curves, starspot evolution was not introduced. However, in real systems starspot evolution would exist. This means that our program can give information about overall trends in starspot evolution, but it cannot be trusted to give the exact brightness values of a region for any given time step. 

In all cases, the light curve fits look good for synthetic curves and real data alike. The fit of the light curve is good irrespective of noise level in the synthetic curve.

To visualize the brightness values, we use Figures~\ref{stripe_plot} and~\ref{box_plot}. The value on the far right is the input value for the synthetic light curves. The y-axis in these images represents longitude along the stellar surface.

\begin{figure}[h]
	\centering
	\includegraphics[width=.5\textwidth]{images/2b_2s/14_noise/box_plot.png}
	\caption{Recovered stripe brightness plot versus window number. The color bar (right) shows the relative brightness value relations to the color scale.}
	\label{box_plot}
\end{figure}
\begin{figure}[h]
	\centering
	\includegraphics[width=.5\textwidth]{images/2b_2s/14_noise/stripe_plot.png}
	\caption{Recovered stripe brightness plot versus window number. The color bar (right) shows the relative brightness value relations to the color scale.}
	\label{stripe_plot}
\end{figure}
\begin{figure}[h]
	\centering
	\includegraphics[width=.5\textwidth]{images/1b/14_noise/stripe_plot.png}
	\caption{Recovered stripe brightness plot versus window number. The color bar (right) shows the relative brightness value relations to the color scale.}
	\label{stripe_plot14}
\end{figure}
\begin{figure}[h]
	\centering
	\includegraphics[width=.5\textwidth]{images/1b/noise/stripe_plot.png}
	\caption{Recovered stripe brightness plot versus window number. The color bar (right) shows the relative brightness value relations to the color scale.}
	\label{stripe_plot12}
\end{figure}
\begin{figure}[h]
	\centering
	\includegraphics[width=.5\textwidth]{images/1b/no_noise/stripe_plot.png}
	\caption{Recovered stripe brightness plot versus window number. The color bar (right) shows the relative brightness value relations to the color scale.}
	\label{stripe_plot}
\end{figure}
\begin{figure}[h]
	\centering
	\includegraphics[width=.5\textwidth]{images/1b/14_noise/box_plot.png}
	\caption{Recovered box brightness plot versus window number. The color bar (right) shows the relative brightness value relations to the color scale.}
	\label{box_plot14}
\end{figure}
\begin{figure}[h]
	\centering
	\includegraphics[width=.5\textwidth]{images/1b/noise/box_plot.png}
	\caption{Recovered box brightness plot versus window number. The color bar (right) shows the relative brightness value relations to the color scale.}
	\label{box_plot12}
\end{figure}
\begin{figure}[h]
	\centering
	\includegraphics[width=.5\textwidth]{images/1b/no_noise/box_plot.png}
	\caption{Recovered box brightness plot versus window number. The color bar (right) shows the relative brightness value relations to the color scale.}
	\label{box_plot}
\end{figure}
\begin{figure}[h]
	\centering
	\includegraphics[width=.5\textwidth]{images/1b/14_noise/rms_over_time.png}
	\caption{RMS between the brightness values used to produce the synthetic curves and the brightness values recovered by the Amoeba Algorithm.}
	\label{rms14}
\end{figure}
\begin{figure}[h]
	\centering
	\includegraphics[width=.5\textwidth]{images/1b/noise/rms_over_time.png}
	\caption{RMS between the brightness values used to produce the synthetic curves and the brightness values recovered by the Amoeba Algorithm.}
	\label{rms12}
\end{figure}
\begin{figure}[h]
	\centering
	\includegraphics[width=.5\textwidth]{images/1b/no_noise/rms_over_time.png}
	\caption{RMS between the brightness values used to produce the synthetic curves and the brightness values recovered by the Amoeba Algorithm.}
	\label{rms}
\end{figure}

This particular set of images shows the brightness recovery for a synthetic light curve produced by darkening two stripe regions and two box regions and simulated 14th magnitude Gaussian noise. The code does a good job of recovering the brightness values for both boxes and stripes. The stripes, in general, are recovered more of the time, more precisely, and more accurately than the boxes. At any given window, one can usually believe a particular stripe measurement. The resulting stellar brightness map from this same set of recovered brightness values is shown in Figure~\ref{bright_map}.


%Figures for the text
%1b_1s
%3b_2s
%all_b_2s
\vspace{9mm}
\appendix

\section{Visibility calculations \label{vis_appendix}}
*********Details of the vis preamble

\subsection{Basic integration for visibility calculations}
Recall equation~\ref{vis_equation} from Section~\ref{visibility}. In this equation, the quadratic limb darkening law is taken directly from \citet{Claret2004}:
\begin{equation}
   \frac{I(\mu)}{I(0)} = 1 - c_1 (1 - \mu) - c_2 (1 - \mu)^2
\end{equation}

Where $\mu = \sin\theta\cos\phi$ in spherical coordinates. Substituting equation~\ref{ld} into equation~\ref{vis_equation} yields:

\begin{equation}
\begin{split}
    V_{i,j} = \int_{\phi_1}^{\phi_2}  \int_{\theta_1}^{\theta_2} (1 - c_1 - c_2) \sin^2 \theta \cos \phi &+ (c_1 + 2 c_2) \sin^3 \theta \cos^2 \phi - c_2 \sin^4 \theta \cos^3 \phi \,\mathrm{d}\theta \, \mathrm{d}\phi
\end{split}
\end{equation}

We calculate this integral at every timestep for every region with $\phi_1$ and $\phi_2$ changing based on the region number and the rotational phase. $\theta_1$ and $\theta_2$ are constant in time, but different for boxes and longitudes as described in Section~\ref{visibilities}.

%This can be split into three separate integrals each with different degrees of $\sin\theta\cos\phi$. 
%These three answers are given below:
%
%\begin{equation}
%\begin{split}
%    V_{i,j}^0 =& (1 - c_1 - c_2) \frac{(\sin\phi_2 - \sin\phi_1)(\theta_2 - \theta_1 - \cos\theta_2\sin\theta_2 + \cos\theta_1\sin\theta_1)}{2}
%\end{split}
%\end{equation}
%
%\begin{equation}
%\begin{split}
%    V_{i,j}^1 =& (c_1 + 2 c_2) \frac{(9(\cos\theta_1 - \cos\theta_2) + \cos(3\theta_2) - \cos(3\theta_1))(\phi_2 - \phi_1 + \cos\phi_2\sin\phi_2 - \cos\phi_1\sin\phi_1)}{24}
%\end{split}
%\end{equation}
%
%\begin{equation}
%\begin{split}
%    V_{i,j}^2 =& - c_2 \frac{(12(\theta_2 - \theta_1) + 8(\sin(2\theta_1) - \sin(2\theta_2)) + \sin(4\theta_2) - \sin(4\theta_1))(9(\sin\phi_2 - \sin\phi_1) + \sin(3\phi_2) - \sin(3\phi_1))}{384}
%\end{split}
%\end{equation}
%
%Thus, the total visibility for a region $j$ at time $i$ is given by $V_{i,j}^0 + V_{i,j}^1 + V_{i,j}^2$.

\vspace{9mm}

\subsection{The transiting planet calculations \label{trans_appendix}}
The simplest case in which the planet is entirely contained within a box is:
\begin{equation}
	A_{occluded} = \pi r_p^2
\end{equation}

The planet can also be partially contained in a box in many ways, shown in Table~\ref{cases}. The sliver of the planet that is not within the box will be calculated and then extended to take care of all of the various cases. To do this, the area of a sector of a circle is found and then the triangle that is formed within it is subtracted. The sector is the combination of the light blue and yellow regions in Figure~\ref{eclipse}.
*******Make this eps
\begin{figure}[h]
	\centering
	\includegraphics[width=.5\textwidth]{images/figure.png}
	\caption{Geometry of eclipse path visibility.}
	\label{eclipse}
\end{figure}

\begin{equation}
	A_{sector} = \frac{r_p \alpha^2}{2}
\end{equation}

where:

\begin{equation}
	\alpha = 2 \cos^{-1}\left(\frac{r^{\prime}}{r_p}\right)
\end{equation}

The area of the triangle is $cr^{\prime}$. $c$ is one half the length of the chord in the figure. $c$ can be found with the Pythagorean theorem. Knowing this, the area of the triangle becomes: 

\begin{equation}
	A_{triangle} = r^{\prime}\sqrt{r_p^2 - {r^{\prime}}^2}
\end{equation}

Finally, the area of the segment that indicates the part of the planet just over the border of the box is:

\begin{equation}
	A_{seg} = r_p^2 \cos^{-1}\left(\frac{r^{\prime}}{r_p}\right) - r^{\prime} \sqrt{r_p^2 - {r^{\prime}}^2}
\end{equation}

After completing the calculation of the area eclipsed by the planet, we multiply it by the average limb-darkened intensity in the containing box.


\begin{table}
	\caption{Subcases for planetary eclipse visibility}
	\label{cases}
	\begin{center}
	\renewcommand{\arraystretch}{1.2}
		\begin{tabular}{| m{.04\textwidth} | m{.24\textwidth} | m{.15\textwidth} |} %c means center justify, l left, r right
			\hline
			\textbf{Case}    & \textbf{Description} & \textbf{Area covered by planet in box}\\ %Separate with &, lines with \\
			\hline%horizontal line
			I      &   Planet is completely out of the box to the right                                                                      & 0                                                           \\ \hline
			II     &   Planet is completely out of the box to the left                                                                         & 0                                                           \\ \hline
			III    &   Planet is completely contained in the box                                                                              & $\pi r_p^2$                                         \\ \hline
			IV    &   Planet is partially off the right side of the box. Center is inside of the box                       & $\pi r_p^2 - A_{seg,l}$                     \\ \hline
			V     &   Planet is partially off the right side of the box. Center is outside of the box                     & $A_{seg}$                                          \\ \hline
			VI    &   Planet is partially off the left side of the box. Center is inside of the box                          & $\pi r_p^2 - A_{seg,r}$                     \\ \hline
			VII   &   Planet is partially off the left side of the box. Center is outside of the box                        & $A_{seg}$                                          \\ \hline
			VIII  &   Planet is partially off both sides of the box. Center is inside of the box                            & $\pi r_p^2 - A_{seg,r} - A_{seg,l}$ \\ \hline
			IX    &   Planet is partially off the right side of the box. Center is outside of the box to the left & $A_{seg,l1} - A_{seg,l2}$                \\ \hline
			X     &   Planet is partially off the left side of the box. Center is outside of the box to the right    & $A_{seg,r1} - A_{seg,r2}$               \\ \hline
		\end{tabular}
	\end{center}
\end{table}

\vspace{9mm}
\section{Results Continued \label{results_appendix}}
Using $A_{seg} $ and Table~\ref{cases}, a believable box-car transit model is produced. Note that $A_{seg,l}$ refers to a segment that is off the left side of a box relative to the center of the planet and likewise for $A_{seg,r}$. The subscripts $A_{seg,r1}$ and $A_{seg,r2}$ and likewise for the left refer to cases where there are two portions off one side of a box relative to the center of the planet. Limb darkening must also be included in order to give a better approximation to the actual shape of a real transit. This is introduced in the visibility rather than in the model flux calculation. This is okay because $\fmod = V_{i,j} b{j}$. If limb-darkening were calculated during every flux calculation of every chi-squared call during the Amoeba algorithm run, it would require much more operational complexity in the code. However, limb-darkening can be calculated before the many chi-squared calls via the Amoeba algorithm and and avoid this.  The geometric analysis of the transit and limb darkening is continued in the next section.


The top two plots show the brightness map as it appears on a projected star. The left map shows the input brightness values to the system for the production of the light curves. The map on the right shows the average recovered brightness over 24 windows that were modeled over the 30 day data set. Each window was 14 days wide and the increment between start times of the windows was 0.3 days. The next two plots show the box and stripe brightness values as a function of window. The region number corresponds to the tuple defined in Section~\ref{model_flux}, Equation~\ref{tuple}. The hue represents the brightness value and should be the same scale for all four of these pictures reporting brightness values. Each column in these plots is one window. The far right column (separated by a line) shows the input brightness value for that region. The next plot shows the standard deviation of the average recovered value in a region compared to the input value for that region. The boxes are to the left of the black line, and the stripes are to the right. Note that the region numbers are the same as in the box and stripe brightness plots above. The final plot is the fit to the synthetic light curve. This plot shows the entire month of data that was created. Every model is over-plotted. The model light curves are shown in black while the synthetic light curves are the red points. Note that models overlap significantly due to the fact that the start times are only 0.3 days (~$\frac{1}{5}$ orbital phase units) apart per window while the window length is 14 days.

\clearpage
\begin{figure}
	\includegraphics[width=1\textwidth]{images/1b_14_page.eps}
	\caption{Diagnostic plots for synthetic starspot system recovery.}
	\label{page_1b}
\end{figure}
\clearpage
\begin{figure}
	\includegraphics[width=1\textwidth]{images/1b_14_transit.eps}
	\caption{Each of the transits for the month of synthetic data. The transits are in order as English is read. Each transit plot shows the synthetic data in red points with every relevant model overlaid in grey lines.}
	\label{transits_1b}
\end{figure}

\clearpage
\begin{figure}
	\includegraphics[width=1\textwidth]{images/1b_1s_14_page.eps}
	\caption{Diagnostic plots for synthetic starspot system recoverys.}
	\label{page_1b_1s}
\end{figure}
\clearpage
\begin{figure}
	\includegraphics[width=1\textwidth]{images/1b_1s_14_transit.eps}
	\caption{Each of the transits for the month of synthetic data. The transits are in order as English is read. Each transit plot shows the synthetic data in red points with every relevant model overlaid in grey lines.}
	\label{transits_1b_1s}
\end{figure}

\clearpage
\begin{figure}
	\includegraphics[width=1\textwidth]{images/2b_14_page.eps}
	\caption{Diagnostic plots for synthetic starspot system recovery.}
	\label{page_2b}
\end{figure}
\clearpage
\begin{figure}
	\includegraphics[width=1\textwidth]{images/2b_14_transit.eps}
	\caption{Each of the transits for the month of synthetic data. The transits are in order as English is read. Each transit plot shows the synthetic data in red points with every relevant model overlaid in grey lines.}
	\label{transits_2b}
\end{figure}

\clearpage
\begin{figure}
	\includegraphics[width=1\textwidth]{images/2b_1s_14_page.eps}
	\caption{Diagnostic plots for synthetic starspot system recovery.}
	\label{page_2b_1s}
\end{figure}
\clearpage
\begin{figure}
	\includegraphics[width=1\textwidth]{images/2b_1s_14_transit.eps}
	\caption{Each of the transits for the month of synthetic data. The transits are in order as English is read. Each transit plot shows the synthetic data in red points with every relevant model overlaid in grey lines.}
	\label{transits_2b_1s}
\end{figure}

\clearpage
\begin{figure}
	\includegraphics[width=1\textwidth]{images/2b_2s_14_page.eps}
	\caption{Diagnostic plots for synthetic starspot system recovery..}
	\label{page_2b_2s}
\end{figure}
\clearpage
\begin{figure}
	\includegraphics[width=1\textwidth]{images/2b_2s_14_transit.eps}
	\caption{Each of the transits for the month of synthetic data. The transits are in order as English is read. Each transit plot shows the synthetic data in red points with every relevant model overlaid in grey lines.}
	\label{transits_2b_2s}
\end{figure}

\clearpage
\begin{figure}
	\includegraphics[width=1\textwidth]{images/3b_1s_14_page.eps}
	\caption{Diagnostic plots for synthetic starspot system recovery.}
	\label{page_3b_1s}
\end{figure}
\clearpage
\begin{figure}
	\includegraphics[width=1\textwidth]{images/3b_1s_14_transit.eps}
	\caption{Each of the transits for the month of synthetic data. The transits are in order as English is read. Each transit plot shows the synthetic data in red points with every relevant model overlaid in grey lines.}
	\label{transits_3b_1s}
\end{figure}


\small
\clearpage
\bibliography{bibliography_astro}{}

\end{document}
